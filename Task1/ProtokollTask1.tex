\NeedsTeXFormat{LaTeX2e}
\documentclass[a4paper,12pt,
headsepline,           % Linie zw. Kopfzeile und Text
oneside,               % zweiseitig
pointlessnumbers,      % keine Punkte nach den letzten Ziffern in Überschriften
bibtotoc,              % LV im IV
%DIV=15,               % Satzspiegel auf 15er Raster, schmalere Ränder   
BCOR15mm               % Bindekorrektur
%,draft
]{scrbook}
\KOMAoptions{DIV=last} % Neuberechnung Satzspiegel nach Laden von Paket helvet

\pagestyle{headings}
\usepackage{blindtext}

% für Texte in deutscher Sprache
\usepackage[ngerman]{babel}
\usepackage[utf8]{inputenc}
\usepackage[T1]{fontenc}

% Helvetica als Standard-Dokumentschrift
\usepackage[scaled]{helvet}
%\usepackage{mathptmx}
\renewcommand{\familydefault}{\sfdefault} 

\usepackage{graphicx}

% Für Tabellen mit fester Gesamtbreite und variabler Spaltenbreite
\usepackage{tabularx} 

% Besondere Schriftauszeichnungen
\usepackage{url}              % \url{http://...} in Schreibmaschinenschrift
\usepackage{color}            % zum Setzen farbigen Textes

\usepackage{amssymb, amsmath} % Pakete für Mathe-Umgebungen und -Symbole

\usepackage{setspace}         % Paket für div. Abstände, z.B. ZA
%\onehalfspacing              % nur dann, wenn gefordert; ist sehr groß!!
\setlength{\parindent}{0pt}   % kein linker Einzug der ersten Absatzzeile
\setlength{\parskip}{1.4ex plus 0.35ex minus 0.3ex} % Absatzabstand, leicht variabel

% Tiefe, bis zu der Überschriften in das Inhaltsverzeichnis kommen
\setcounter{tocdepth}{3}      % ist Standard


%\usepackage[printonlyused, withpage]{acronym}
\usepackage[printonlyused]{acronym}
\usepackage{comment}
\usepackage{longtable}
%\usepackage[maxdepth=8, mark]{gitinfo2}
\usepackage{bytefield}
\usepackage{smartdiagram}
\usepackage{caption}
\usepackage{hyperref}
\usepackage{hypcap}
\usepackage{graphicx}
\usepackage{tikz}
\usepackage{mathtools}
\usepackage{colortbl}
\usepackage{xcolor}
\usepackage[outline]{contour}
\usetikzlibrary{arrows}
\usetikzlibrary{calc,positioning,shapes,decorations.pathreplacing, matrix, backgrounds, fit}

% hier Namen etc. einsetzen
\newcommand{\fullname}{Björn Petersen}
\newcommand{\email}{bjoern.petersen@uni-ulm.de}
\newcommand{\titel}{Darstellung und Analyse SoC interner Kommunikationskanäle auf Protokollebene}
\newcommand{\jahr}{2020}
\newcommand{\matnr}{941984}
\newcommand{\gutachterA}{Prof.\,Dr.-Ing\,Maurits Ortmanns}
\newcommand{\gutachterB}{Dr.\,rer.\,nat.\,Dipl.-Ing\,Endric Schubert}
\newcommand{\betreuer}{Dipl-Ing\, Ulrich Langenbach}

% hier die Fakultät auswählen
%\newcommand{\fakultaet}{---  Im Quellcode anpassen nicht vergessen! ---}
\newcommand{\fakultaet}{Ingenieurwissenschaften, Informatik und\\Psychologie}
%\newcommand{\fakultaet}{Mathematik und\\Wirtschafts-\\wissenschaften}
%\newcommand{\fakultaet}{Medizin}
%\newcommand{\fakultaet}{Naturwissenschaften}

% hier das Institut einsetzen
\newcommand{\institut}{Institut für Mikroelektronik}

% Informationen, die LaTeX in die PDF-Datei schreibt
\pdfinfo{
  /Author (\fullname)
  /Title (\titel)
  /Producer     (pdfeTex 3.14159-1.30.6-2.2)
  /Keywords ()
}

\usepackage{hyperref}
\hypersetup{
pdftitle=\titel,
pdfauthor=\fullname,
pdfsubject={Diplomarbeit},
pdfproducer={pdfeTex 3.14159-1.30.6-2.2},
colorlinks=false,
pdfborder=0 0 0	% keine Box um die Links!
}

\usepackage{listings}
\lstset{
  language=bash,
  basicstyle=\ttfamily,
  columns=fullflexible,
  frame=single,
  breaklines=true,
  postbreak=\mbox{\textcolor{red}{$\hookrightarrow$}\space}
}

% Trennungsregeln
\hyphenation{Sil-ben-trenn-ung}

\title{Modern Port- and Security Scanning}
\date{Juli 20, 2020}
\author{Björn Petersen}

\begin{document}

\maketitle

\section*{Überprüfen des Setups}

\subsection*{Docker}
\lstinputlisting{docker.bash}

\subsection*{ZMap}
\lstinputlisting{zmap.bash}

\subsection*{Interfaces}
\lstinputlisting{interfaces.bash}

\section*{Host Discovery}
\subsection*{Aufgabenteil a}
\lstinputlisting{hostDiscovery.bash}

\subsection*{Aufgabenteil b}
\subsubsection*{Standardpakete}
NMap nutzt standardmäßig für die Porterkennung:
\begin{itemize}
  \item Ein ICMP-Echo-Request
  \item Ein TCP-SYN an Port 443
  \item Ein TCP-ACk an Port 80
  \item Ein ICMP-Timestamp-Request Informationen dazu aus der man-page:
\end{itemize}

\subsubsection*{Unterschiede Priviligierter User}
Die meisten Scan-Typen stehen nur privilegierten Benutzern zur Verfügung, weil sie rohe IP-Pakete senden und empfangen, wofür auf Unix-Systemen root-Rechte benötigt werden. 

\subsubsection*{Private/ Öffentliche Netze}
Bei öffentlichen Netzen ist oft nur ein Gerät sichtbar, da die anderen Geräte hinter einem NAT stecken. Andernfalls gibt es technisch keinen Unterschied, außer das öffentliche Netze natürlich rechtlich kritischer sein können.

\subsection*{Aufgabenteil c}
\lstinputlisting{scanme.bash}

Es lässt sich sehen, dass als OS Linux verwendet wird, sowie  ein SSH und ein HTTP Server laufen. Des Weiteren laufen auf Port 9929 und 31337 ein TCP Server.

\section*{Genaueres Untersuchen des Hosts}
\subsection*{Aufgabenteil a}
\lstinputlisting{portscan.bash}
\subsection*{Aufgabenteil b}
\subsubsection*{I}
Das Betriebssystem des Hosts könnte auch bei geschlossenen Ports mit einem SYN reagieren
\subsubsection*{II}
Die ports könnten von einer Firewall blockiert werden
\subsubsection*{III}
Man müsste die Firewall umgehen können, was beispielsweise mit einem Proxy möglich wäre
\subsubsection*{IV}
keine Standartisierten Portnummer, Firewall

\subsection*{Aufgabenteil c}
\begin{itemize}
  \item -sU (UDP Scan): Funktioniert nur bei UDP
  \item -sA (TCP-ACK Scan): Ermittelt ob Port von einer Firewall blockiert ist, nicht ob er offen ist
  \item -sS (SYN Scan): Braucht Root Recht, geht nur bei TCP
  \item -sF (FIN Scan): Braucht Root Rechte, nutzt aber aus, dass geschlossene Ports auf FIN mit RST antworten
\end{itemize}

\subsubsection*{UDP Scan}
\lstinputlisting{udpPortscan.bash}

\subsubsection*{TCP-ACK Scan}
\lstinputlisting{achPortScan.bash}

\subsubsection*{SYN Scan}
siehe oben.

\subsubsection*{FIN Scan}
\lstinputlisting{finPortScan.bash}

\subsubsection*{Unterschiede der Varianten}

Beim ACK-Scan lässt sich für alle Hosts herausfinden, ob sie von einer Firewall geschützt werden, des weiteren lässt sich sehen, dass beim FIN-Scan und ACK-Scan nicht alle Ports als offen gezeigt werden.

\subsection*{Aufgabenteil d}
\lstinputlisting{schwachstellen.bash}

Es lässt sich erkennen, das FTP Server läuft, welcher Klartext annimmt und für einen FTP-Bounce Scan genutzt werden könnte. Es läuft ein SMTP Server, der aber eher kein Risiko darstellen sollte, da hier kein E-Mail Verkehr zu erwarten ist. Der SSH Server bietet ebenfalls wenig Angrifsfläche. Dahingegen läuft
ein Telnet-Server, welcher direkten Zugriff gewährt und somit ein Risiko bietet. Auf Port 1524 läuft ein Ingelsrock Server, welcher schon direkt beim scan ein Teil des Betriebssystems anzeigt
Der Apache und MySQL Server könnten ebenfalls weitere Schwachstellen darstellen

\section*{Firewall Evasion}
Als erstes wird die Firewall genauer angeschaut durch einen TCP-ACk scan
\lstinputlisting{firewallEvasion.bash}
Durch mehrfaches wiederholen fällt auf, dass sich die Ports ändern. Durch ausprobieren mehrerer Portnummern auf gut Glück hat man irgendwann erfolg
\lstinputlisting{portTryingWebServer.bash}
Auf den Host aus der vorigen Aufgabe kann über Telnet zugegriffen werden
\lstinputlisting{telnetLogin.bash}
Über Telnet lässt sich nun ein Netzwerk Scan durchführen
\lstinputlisting{telnetScan.bash}
Es lässt sich erkennen, dass bei 172.20.1.10 alle Ports offen sind, bei 11 keine und bei 13 nur Port 53. Stellt man nun mit curl eine Anfrage an den Webserver auf Port 53 erhält man eine Antwort
\lstinputlisting{telnetHTTPAnfrage.bash}
Auf 172.20.1.10 lässt sich ebenfalls der Webserver durch ausprobieren finden
\lstinputlisting{telnetHTTPAnfrage2.bash}

\section*{Nmap vs Zmap}
\subsection*{Aufgabenteil a}
Der IP-Block der Uni ist 134.60.0.0/16
\subsection*{Aufgabenteil b}
\lstinputlisting{nmapUniScan.bash}
Nmap findet 98 Hosts in 48.76 Sekunden
\subsection*{Aufgabenteil c}
\lstinputlisting{zmap.bash}
Zmap ist wesentlich schneller, findet aber weniger Geräte. Dies liegt daran, dass Zmap die Scans variiert.
\subsection*{Aufgabenteil d}
\lstinputlisting{nmapUniOffenePorts.bash}
Werden nur Geräte mit offenem Port 80 gescannt, ist Nmap ähnlich schnell wie Zmap

\section*{Inventarisierung des Uninetzes}
\subsection*{Aufgabenteil a}
Zum Zeitpunkt des Scannens wurde ein Gerät mit offenem Port 443 gefunden
\lstinputlisting{httpsServer.bash}

\subsection*{Aufgabenteil b, c, d}
Leider habe ich bei der Suche nach den Apache Servern oder SSL kein einziges Gerät finden können, weshalb ich keine Ausgabe habe. Bei der Suche nach dem Betriebssystem habe ich nmap nach 15 Minuten ohne Ausgabe abgebrochen, wodurch ich die Teilaufgabe nicht bearbeiten konnte

\end{document}
